\section{Spark Integration with HBase: Representing Log Structured Merge Trees
with a Directed Acyclic Graph}
The merging of an LSM tree (immutable data structures) with a Directed Acyclic
Graph should yield significant performance capabilities over simply rescanning
all the data out of the LSM tree for each execution.  

\subsection{Splice Explain Plan Transformed from LSM to Performant DAG
Structure} This section will go through a simple splice explain plan and then
view it in the current DAG structure and the proposed DAG structure.

\subsubsection{Splice Explain Tree}
Here is the explain plan for a simple 3 way inner join.

\tikzstyle{abstract}=[rectangle, draw=black, rounded corners, fill=blue!20, drop
shadow, text centered, anchor=north, text=white, text width=3cm]
\tikzstyle{comment}=[rectangle, draw=black, rounded corners, fill=green, drop shadow,
        text centered, anchor=north, text=white, text width=3cm]
\tikzstyle{myarrow}=[->, >=open triangle 90, thick]
\tikzstyle{line}=[-, thick]
        
\begin{center}
\begin{tikzpicture}[node distance=0.4cm]
    \node (Table1) [abstract, rectangle split, rectangle split parts=1] {
            \textbf{$Table_{1}$}
        };
    \node (Join1) [abstract, rectangle split, rectangle split parts=1,    
    above right=of Table1] {
            \textbf{$Join_{1}$}
        };
    \node (Table2) [abstract, rectangle split, rectangle split
    parts=1,below right=of Join1] {
            \textbf{$Table_{2}$}
        };
    \node (Projection1) [abstract, rectangle split, rectangle split parts=1,    
        above right=of Join1] {
                    \textbf{$Projection_{1}$}
        };
    \node (Join2) [abstract, rectangle split, rectangle split parts=1,    
        above right=of Projection1] {
            \textbf{$Join_{2}$}
        };
    \node (Table3) [abstract, rectangle split, rectangle split parts=1,    
        below right=of Join2] {
        \textbf{$Table_{3}$}
        };
        
         \path (Table1) edge (Join1);
         \path (Join1) edge (Projection1);
         \path (Join1) edge (Table2);
         \path (Projection1) edge (Join2);
         \path (Join2) edge (Table3);
\end{tikzpicture}
\end{center}

\subsubsection{Current DAG Model}
Here is the current DAG model based on the current thinking on HBase integration
with Spark via Hadoop Input Formats.

\tikzstyle{abstract}=[rectangle, draw=black, rounded corners, fill=blue!20, drop
shadow, text centered, anchor=north, text=white, text width=3cm]
\tikzstyle{comment}=[rectangle, draw=black, rounded corners, fill=green, drop shadow,
        text centered, anchor=north, text=white, text width=3cm]
\tikzstyle{myarrow}=[->, >=open triangle 90, thick]
\tikzstyle{line}=[-, thick]
        
\begin{center}
\begin{tikzpicture}[node distance=0.4cm]
    \node (Table1) [abstract, rectangle split, rectangle split parts=2] {
            \textbf{$Table_{1}$}
            \nodepart{second}Region Based RDD
        };
    \node (Join1) [abstract, rectangle split, rectangle split parts=2,    
    above right=of Table1] {
            \textbf{$Join_{1}$}
            \nodepart{second}Post Join RDD
        };
    \node (Table2) [abstract, rectangle split, rectangle split
    parts=2,below right=of Join1] {
            \textbf{$Table_{2}$}
            \nodepart{second}Region Based RDD
        };
    \node (Projection1) [abstract, rectangle split, rectangle split parts=1,    
        above right=of Join1] {
                    \textbf{$Projection_{1}$}
        };
    \node (Join2) [abstract, rectangle split, rectangle split parts=2,    
        above right=of Projection1] {
            \textbf{$Join_{2}$}
            \nodepart{second}Post Join RDD
        };
    \node (Table3) [abstract, rectangle split, rectangle split parts=2,    
        below right=of Join2] {
        \textbf{$Table_{3}$}
        \nodepart{second}Region Based RDD
        };
        
         \path (Table1) edge (Join1);
         \path (Join1) edge (Projection1);
         \path (Join1) edge (Table2);
         \path (Projection1) edge (Join2);
         \path (Join2) edge (Table3);


%        
%        
        
\end{tikzpicture}
\end{center}

This approach has several significant pitfalls listed below:

\begin{enumerate}
	\item Scan Based RDD: Each individual scan against the data constructs a
	different RDD based on the regions, filters, and fields participating.
	\item Data Movement: The entire HBase results for the table scans must be
	externally scanned (scan local to node but still external to process) and
	transferred to the DAG model during each execution.  This causes all members of
	the graph to be impossible to reuse.
	\item Snapshot Isolation Application: Since Snapshot Isolation is applied in
	HBase, each read from the HBase table could have a slightly smaller 
	\item Updates:  A single update to a region will cause the DAG model to re-read
	all regions for that table.  For example, you have 1TB of data and a single
	update will require you to re-read the 1TB of data.
\end{enumerate}

\subsubsection{Proposed Simple DAG Model}
Here is the proposed Simple DAG model based on trying to limit the amount of
data needed to be read from HBase.  The changes are centered around the
interaction with HBase Tables.  The remainder of the plan tree will be discarded
for this analysis.

\tikzstyle{abstract}=[rectangle, draw=black, rounded corners, fill=blue!20, drop
shadow, text centered, anchor=north, text=white, text width=3cm]
\tikzstyle{comment}=[rectangle, draw=black, rounded corners, fill=green, drop shadow,
        text centered, anchor=north, text=white, text width=3cm]
\tikzstyle{myarrow}=[->, >=open triangle 90, thick]
\tikzstyle{line}=[-, thick]
        
\begin{center}
\begin{tikzpicture}[node distance=0.4cm]
    \node (Table1) [abstract, rectangle split, rectangle split parts=2] {
            \textbf{$Table_{1}$}
            \nodepart{second}No RDD, Just Scan Logic
        };

    \node (Table1Region1) [abstract, rectangle split, rectangle split parts=2,    
    below left=of Table1] {
            \textbf{$Table_{1}$$Region_{1}$}
            \nodepart{second}RDD
        };

    \node (Table1RegionN) [abstract, rectangle split, rectangle split parts=2,    
    below right=of Table1] {
            \textbf{$Table_{1}$$Region_{n}$}
            \nodepart{second}RDD
        };

        
         \path (Table1RegionN) edge (Table1);
         \path (Table1Region1) edge (Table1);

\end{tikzpicture}
\end{center}

\begin{enumerate}
	\item Scan Based RDD: Scans will go against the same cached RDD for evaluation.
	\item Data Movement: Only regions with updated data will invalidate the RDD and
	force a reload.
	\item Snapshot Isolation Application: ? (XXX - TODO JL)
	\item Updates:  A single update to a region will cause the DAG model to re-read
	that single regions for that table (1 Gig).  For example, you have 1TB of data
	and a single update will require you to re-read only 1GB of data.
\end{enumerate}
        

%End LSM to DAG chapter
