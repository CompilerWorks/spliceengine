\newcommand{\collecttablestats}{\texttt{SYSCS\_UTIL.COLLECT\_TABLE\_STATISTICS }}
\newcommand{\collectschemastats}{\texttt{SYSCS\_UTIL.COLLECT\_SCHEMA\_STATISTICS }}
\newcommand{\systablestats}{\texttt{SYS.SYSTABLESTATS }}
\newcommand{\systablestatistics}{\texttt{SYS.SYSTABLESTATISTICS	}}
\newcommand{\syscolumnstats}{\texttt{SYS.SYSCOLUMNSTATS }}
\newcommand{\syscolumnstatistics}{\texttt{SYS.SYSCOLUMNSTATISTICS }}

\section{Overview}
Understanding the role of statistics in a relational database is typically tightly connected to understanding the role of the query optimizer. In most databases\footnote{see Appendix-\ref{OtherDbs}}, statistics information is a tool which is used only during the query planning and optimization stage, and is not used at any other stage of the execution process. This view that statistics is only helpful to the query optimizer has significant consequences on how the statistical systems in those databases have been implemented. There are variations, but the central theme of statistics in database products to this point is to acquire a small by statistically significant sample of data, and computing a simple set of statistics from this sample. This has the advantage of being relatively easy to understand and implement, but tends to destroy any bounds between two individual partitions of data. Because these partition boundaries are destroyed, implementations are then forced into the circular view that statistics is only global in nature.

In most database products, this global view has a minimal impact, because its operations tend to be concentrated into a single operational server anyway\footnote{In the sense that there is typically some form of \emph{master} server which manages the actual query planning}. However, SpliceMachine has a fundamentally different approach. Because each server is able to share both the optimization \emph{and} the execution load of the entire system, we have an essentially distributed architecture. This allows us, with some careful consideration, to approach statistics as both a \emph{global} and a \emph{local} entity--that is, we can either consider statistics for the entire data set as a single, global entity, or we can consider statistics at each individual region. 

There are two significant advantages that this grants us. Firstly, we can perform collections in a different way, as each region of data can be considered to own its data independently of all other regions in a table. Secondly, because each region is isolated, we can use statistics which are collected \emph{for an individual region} to optimize region-specific \emph{execution} resources; in turn, this allows us to ensure that SpliceMachine runs more predictably and stably even under the stress of many simultaneous and distinct workloads.

This document is broken down into 5 sections. In the \textbf{Goals} section, we describe the high-level goals which we are aiming to acheive. These are essentially a restatement of user and product requirements. In the \textbf{Architecture} section, we will describe at a high level how the overall architecture will operate. From there, the \textbf{Collection} section describes how statistics information is to be collected, while the \textbf{Storage} section will describe what will be stored and how. Finally, we provide three significant appendices. \textbf{Appendix-\ref{OtherDBs}} describes statistics systems in other database products.\textbf{Appendix-\ref{Algorithms}} provides mathematical details and justifications for how individual statistics algorithms are to be collected. These appendices provide context about why some decisions were made, and how they affect the overall design of the statistic engine, but are not significant to understanding the design itself.  Finally, \textbf{Appendix-\ref{Future}} provides details about which features are still undeveloped, and slated for future releases. 

\section{Goals}
It seems apparent, but is always worth restating that we do not collect statistics for our health. Indeed, they are critical to the well functioning of a modern database system, in particular when considering a cost-based query optimization engine. However, merely saying "we need statistics" is an insufficient target, we need more precise goals. 

\subsubsection{Support Table-based statistics}
The Statistics module should support table-level statistics, particularly
\begin{enumerate}
\item Number of rows in table
\item Number of partitions in table 
\item Number of Servers managing table
\item Average length of a single row (in bytes)
\item The average latency to read a single row within a single JVM
\item The average latency to read a single row over the network
\end{enumerate}

\subsubsection{Support Column-based statitics}
The module should support column-level statistics, particularly
\begin{enumerate}
\item Cardinality
\item Null count
\item average size of a column (in bytes)
\item the $N$ most frequent elements and their counts (where $N$ is a configurable number).
\end{enumerate}

\subsubsection{Support Index Statistics}
All table and column-level statistics should be collected for indices as well as base conglomerates, as well as the average latency to perform an index lookup for a single row.

\subsubsection{Updating Statistics}
It is important to be able to manually initiate statistics collection, so that administrators are able to manage statistics within their individual workloads. 

\subsubsection{Reporting Statistics}
It is often convenient for administrators to view statistics for individual tables, as rough approximations of the size and distribution of data for a table.

\subsubsection{Automatic collection}
In many use-cases (particularly heavy OLTP workloads), it is difficult to keep statistics up to date manually. Therefore, there must be a mechanism for automatically updating statistics (or a portion of statistics) in an efficient way as needed. Because this can be expensive, disabling it must be possible for workloads which do not wish it.

\subsubsection{Partition-level Statistics collection}
Because statistics on an entire table may require performing IO on the entire table (depending on the implementation), it should be made possible to collect statistics on an individual partition of data as well, in isolation from the other partitions of data.

\subsubsection{Failure Semantics}
Statistics collection failures should fit within the existing failure-reporting system, and not cause catastrophic failures.

\subsubsection{Manage Resources used by Statistics}
Statistics is fundamentally a maintenance operation, and should not interfere with the more important user-driven tasks. Typically, there are three major resources that must be managed:

\begin{enumerate}
\item Disk IO
\item System Memory
\item CPU cycles
\end{enumerate}

The statistics module should be constructed such that administrators are allowed to make tradeoffs between performance of statistics and interruption of other services.



\section{Architecture}
There are three main components of the Statistics engine in SpliceMachine:
\begin{enumerate}
\item[Collection] approaches what statistics are collected, and how that collection will occur
\item[Storage] resolves issues around storing statistical data so that it may be efficiently accessed, and human visibility
\item[Access] describes the structure of accessing statistical data quickly for internal systems
\end{enumerate}

\subsection{Collection} 
Statistics are collected through the use of a periodically executed maintenance tasks, which can be triggered on a single region in one of two modes. Manual mode will collect statistics when the administrator issues the appropriate request. There will be some variations on those procedures to ensure that statistics can be collected with varying degrees of thoroughness (and with varying performance characteristics); for example, statistics may be collected on individual tables and schemas.

Automatic mode, by contrast, will attempt to refresh statistics whenever the system can reasonably detect that new statistics be helpful. In particular, whenever a region detects that it has received a significant number of successful writes, it will assume that it needs new statistics, and will submit a task for execution. However, this maintenance task will wait to be executed until the region detects a substantial decrease in it's overall write load (Should such a situation never occur, it will eventually execute anyway).

As automatic mode may choose to use resources at an inopurtune time, it will come with an associated disable call, which an administrator can use to disable automatic collection for high-load databases.  

\subsection{Storage}
Once collected for a given partition, Statistics must be made available for the query optimizers on \emph{all} nodes to use. 

The most obvious way to do this is to store the data into an HBase table. In particular, there will be two tables: 

\begin{description}
				\item[\texttt{systablestats}] maintains statistics for the table as a whole, including latency and other physical statistics.
				\item[\texttt{syscolumnstats}] maintains statistics for each individual column
\end{description}

Each partition will update a single row in each table, making the storage format partition-specific. This allows individual partitions to update themselves without requiring the entire table to update.

These two tables will hold data in binary formats which are efficient to store and can be combined correctly, but will not be human readable in any direct sense. To allow human visibility into the system, we will also create two read-only views:

\begin{description}
				\item[\texttt{systablestatistics}] is a human-readable view of the table as a whole
				\item[\texttt{syscolumnstatistics}] is a human-readable view of for each individual column
\end{description}
Only the statistics engine itself will be able to modify these views.

\subsection{Access}
There is a significant downside to this strategy that we must deal with. We note that table access is a relatively expensive operation to conduct remotely, and even more expensive when that data is stored in only a single region (which it is reasonable to expect, since a single cluster will have thousands of regions at most). However, we are helped by the realization that statistics updates will occur relatively rarely in our system\footnote{Rarely in this case can be anywhere between once a month for a reference data set to once every hour for high-volume OLTP tables}. Because of this, we can safely cache statistics data locally on each RegionServer on the cluster. We refer to this cache as the \emph{PartitionCache}, which contains a cached version of statistics both as a global view and as a partition-specific view. 

We internally represent statistics as a global view, where the overall statistic is computed by merging together the results from all known partitions. This merging process is different for each specific algorithm and data structure\footnote{For example, computing the global average is different than averaging the averages}


\section{Statistics}
\label{sec:Statistics}
The strategy that we've outlined so far has expressed itself in the singular "Statistics", but what, exactly, are these statistics that we plan on collecting, and why are they considered useful for us?

Roughly speaking, we would like to collect anything(and everything) that will give us more information about the data that is stored. We are tolerant to some degree of error in our estimates, as long as we can get accurate enough that we make the correct decisions during optimization. 

We can safely group our statistics into two distinct categories: \emph{logical} and \emph{runtime}.  

\subsection{Logical Statistics}
Logical Statistics are what we generally think of when we think of statistics; they involve information that we can use to resolve a (slighty fuzzy) view of the data without requiring reference to the data itself.

We separate our logical statistics into two distinct subcategories. On the global side, we maintain \emph{table}-level statistics, which provide us with information about the table as a whole, ignoring specific columns. For more precise information, we also maintain \emph{column}-level statistics, which provide us with information about the columns themselves.

\subsection{Table-Level Statistics}
Table-level statistics are reasonably simple to maintain and understand, as they deal with rows in the absence of additional column information. 

We maintain the following set of statistics:

\begin{enumerate}
				\item Row Count
				\item Total Size of table in bytes
				\item Mean width of a Row in bytes (including RowKey)
				\item Total Query Count
				\item Region Count
				\item Mean Region Size in bytes
\end{enumerate}

As a note, it is not always strictly necessary to keep all this information stored on disk--for example, the region count can be acquired directly, without reference to any tables.

\subsection{Column Statistics}
Column Statistics manage information about individual columns, and are often the more interesting (and more difficult) algorithms to implement. While table-level metrics are easy to acquire\footnote{In fact, one could simply maintain the correct values for all table-level metrics and have a reasonable degree of accuracy}, column statistics are significantly more complex. 

To start with, we will collect the following statistics:

\begin{enumerate}
				\item Cardinality
				\item \emph{Null Fraction} = number of null elements / total number of rows in table
				\item Frequent Elements
				\item Distribution (if data type is ordered)
				\item Minimum Value (if data type is ordered)
				\item Frequency of Minimum Value
				\item Maximum value (if data type is ordered)
				\item Mean size of a column (in bytes)
\end{enumerate}

These statistics will be stored in an intermediate binary format which is easy to merge together\footnote{in particular, the algorithms which generate statistics are all linear functions}.

\subsubsection{Cardinality}
The first, and simplest, metric that we wish to record is the \emph{cardinality}. Mathematically, \emph{cardinality} is just the number of entries in a set. As we are dealing with multisets(sets which can contain more than one entry with the same value), cardinality is more appropriately stated as the number of \emph{distinct} elements in the data set. 

There are a number of algorithms which can estimate the cardinality of a data set with varying degrees of accuracy. However, the most effective estimation strategy is \hyperref[sec:HyperLogLog]{\emph{HyperLogLog}}\cite{Flajolet07hyperloglog:the}. 

HyperLogLog essentially strikes a balance between memory consumption and accuracy. A more accurate estimate requires more counters,and thus more total memory. However, once a given accuracy is specified, the space requirements are \emph{constant} with respect to the number of rows processed. Additionally, updates to the data structure are constant-time. 

In practice, each counter requires only a byte of space, so the total memory space is the cost of a single object, a single array, and $2^b$ bytes. If $b=14$, this memory cost is $\approx 16$ kilobytes per column(Heule et. al's adjustments can reduce this cost for low cardinalities\cite{HyperLogLogGoogle}). Thus, over the maximum 1024 columns, the total memory footprint due to cardinality checking is $\approx 2^{b+10}$ bytes(for $b=14$, this is $\approx 16$ megabytes).

\subsubsection{Frequent Elements}
The next set of statistics to collect for SpliceMachine is referred to as the \emph{Frequent Elements}\footnote{Also referred to as the \emph{Heavy Hitters}, or the \emph{Iceberg Values}}. Frequent elements are simply the elements which occur most frequently in the data set; collecting them and their exact frequencies allows us to be more accurate for queries in which these frequent elements are involved.

There are fewer algorithms available for solving the Frequent Elements problem, but there are still a reasonably large number to parse through. However, the \hyperref[sec:SpaceSaver]{\emph{SpaceSaver}} algorithm stands out as a particularly effective choice.

The SpaceSaver algorithm keeps an approximate list of heavy hitters using constant space (controlled by the desired accuracy). It does this by keeping not just a fixed-size list of elements, but also a fixed-size list of error estimates, which it can use to eliminate errneously counted heavy hitters\footnote{We elide the details of the algorithm in this section. For those details, see Appendix-\ref{sec:SpaceSaver}}. As a result of the structural designs and a bit of clever data structure organization, SpaceSaver is able to allow constant-time updates with a bounded constant-space memory cost.

Generally speaking, if one wishes to keep the top $N$ most frequent elements, then the SpaceSaver algorithm requires $N$ fields (the item itself), plus $2N$ counters (2 for each item in the data structure. 

\subsection{Physical Statistics}
In addition to logical statistics, we must collect some basic information about the physical world in which we are operating. Because we are durable, a large component of our cost is the cost of reading data off disk; because we are a clustered environment, the second largest power is the cost to write and read data over the network. Thus, we will need to collect disk and network I/O latency. 

while variable measurements are:
\begin{enumerate}
				\item Local Read Latency
				\item Remote Read Latency
				\item Write network latency (to TEMP)
				\item Open Scanner latency
				\item Close Scanner latency
				\item Index Fetch latency (if conglomerate is an index)
\end{enumerate}

These measurements are collected during the logical statistics gathering phase. If the table being measured is an index table, use the same sample to perform a simple index lookup on the main table, which will provide index lookup and write latency measures. Local read measures can be acquired merely by recording the read performance of the statistics gathering process.

When automatic collection is enabled, a good way of obtaining these measurements is similar to that of PostgreSQL\cite{PGCollector} measures real time performance, and is referred to as \emph{query sampling}. When query sampling is enabled, a random sample of queries is taken\footnote{The sampling logic may be selective--for example, it may only only randomly sample from those queries which are expected to take a very short period of time, so that the added cost is not excessive.}. When a query is chosen, it will record the latency measurements that occurred during the execution of that query. This would allow more accurate variable measurements, but would have an adverse impact on performance for selected queries, so it would also require a shut-off valve to disable it when performance is critical.



\section{Collection}
Statistics collection occurs using a manual collection process (there is no automatic collection). There are two main stored procedures which are used to collect and process statistics:

\begin{itemize}
\item \collecttablestats collects statistics for an individual table and its associated indices
\item \collectschemastats collects statistics for all tables and indices in a schema.
\end{itemize}

Both procedures are synchronous--once executed, the procedure will wait until statistics have been collected for every region of every involved table and index. The collection occurs using the task framework, and is not rate-limited (Rate limiting would behave counter to the understanding that a user initiated collection.

A collection for a given conglomerate is a parallel execution where each region involved is \emph{fully scanned}, and data is collected for all rows in that region\footnote{This is because an LSM tree cannot be reasonably sampled}. 

\subsection{Enabling and Disabling Column Statistics}
Collecting statistics is not free, and should not be treated as free:

\begin{enumerate}
\item[Cardinality] requires between $2^{4}$ and $2^{16}$ bytes \emph{per column collected}. If the maximum number of columns ($1024$) is collected with maximum configured accuracy, then this will require $2^{26}$ bytes, or 64 MB of heap.
\item[Frequent Elements] requires $K$ objects and $2K$ longs \emph{per column collected}, where $K$ is the number of frequent elements to keep. If the 100 most frequent elements are kept, then the counters alone require $1.5$ KB \emph{per column}. If the maximum number of columns are collected, then this requires $1.5*1024 \approx 1.5$ MB of heap.
\item[Histograms] are not yet implemented, but when implemented will require $\Omega(\lg{N})$ space for each column (where $N$ is the number of rows in the region).
\end{enumerate}

This is not necessarily a \emph{huge} cost, but it should be considered. Thus, there is a mechanism for enabling and disabling collection for some columns. This can be done by invoking one of the following procedures:

\begin{enumerate}
\item \texttt{SYSCS\_UTIL.ENABLE\_COLUMN\_STATISTICS} 
\item \texttt{SYSCS\_UTIL.DISABLE\_COLUMN\_STATISTICS} 
\end{enumerate}
by default, all columns are collected in Lassen. To determine whether or not a column is enabled for statistics collection, execute
\begin{lstlisting}[frame=single,captionpos=b,caption=Determine if column has statistics enabled,language=SQL]
select
	c.columnname
	,case 
		when c.collectstats is null then true 
		else c.collectstats 
	end as collectstats
from 
	sys.systables t
	,sys.sysschemas s,
 	,sys.syscolumns c
where 
	t.tableid = c.referenceid
	and t.schemaid = s.schemaid
	and t.tablename = '<Table>'
	and s.schemaname = '<Schema>'
\end{lstlisting}

It should be noted that some column types (such as blobs and clobs) cannot be collected, and thus cannot be enabled. It is also worth noting that \emph{keyed} columns\footnote{That is, columns which are included in an index or a primary key} cannot be disabled--they must always be collected. This is because keyed columns are heavily used by the optimizer to ensure that scan selectivity is accurate, and that accuracy is undermined when no statistics are present for a given keyed column.

\subsection{Stale-only collections}
Efficiency dictates that, whenever possible, regions should not be collected unless one can reasonably know that the region needs to be collected. When a region needs a new statistics collection (either because of mutations, or because statistics does not exist), it is said to be \emph{stale}. When collecting statistics for the first time, all regions must be involved, but afterwards only stale regions need to be recollected.

Staleness detection, and the accompanying stale-only collection, is not yet implemented. However, to allow a clean API when stale-only collections are implemented, a boolean \emph{stale-only} parameter is added to the \collecttablestats and \break \collectschemastats stored procedures. This parameter is currently ignored, but is reserved for future use.


\section{Storage}
Statistics data is stored in 3 distinct system tables:
\begin{enumerate}
\item \systablestats
\item \syscolumnstats
\item \texttt{SYS.SYSPHYSICALSTATS}
\end{enumerate}

\texttt{SYS.SYSPHYSICALSTATS} is currently unused--it is reserved for future improvements to the statistics library. 

Statistics are collected and stored on a per-\emph{partition}\footnote{a \emph{partition} is simply a contiguous, uniquely identifiable collection of rows. In Lassen, a partition is the same thing as a region, but there is no reason that must remain true for all time} basis. Because of this, the \systablestats and \syscolumnstats tables are somewhat awkward for human visiblity; one would need to manually aggregate up values to construct a global view of statistics. 

Due to that, we also introduce two major views: 

\begin{enumerate}
\item \systablestatistics
\item \syscolumnstatistics
\end{enumerate}

These views should be treated as the primary mechanism for viewing human-readable statistics information.

\subsection{Table Statistics}
\systablestats stores statistical information about the conglomerate itself; its schema is described in Table-\ref{table:tableStats}. This table is partition-specific. For a human-readable view, use \systablestatistics instead(scheam in Table-\ref{table:tableStatistics}).

\begin{table}
	\begin{tabular}{|l|c|p{6cm}|}
		\hline
		\bf{Name}											&	\bf{Type}	& \bf{Description} \\ \hline
		\texttt{conglomerateid}				&	bigint		& The id of the conglomerate \\ \hline
		\texttt{partitionid}					&	varchar		&	The unique identifier for the partition \\ \hline
		\texttt{last\_updated}				&	timestamp	&	The last time statistics was collected for this conglomerate \\ \hline
		\texttt{is\_stale}						&	boolean		&	Whether or not these statistics are stale \\ \hline
		\texttt{in\_progress}					&	boolean		&	Whether or not a collection is in progress \\ \hline
		\texttt{rowcount}							&	bigint		&	The total number of rows in the partition \\ \hline
		\texttt{partition\_size}			&	bigint		&	The total number of bytes in the partition \\ \hline
		\texttt{meanrowwidth}					&	integer		&	The average width of a row in this partition \\ \hline
		\texttt{querycount}						&	bigint		&	The total number of queries addressed to this region. \\ \hline
		\texttt{localreadlatency}			&	bigint		&	The average time to read a single row within the same JVM (in microseconds) \\ \hline
		\texttt{remotereadlatency}		&	bigint		&	The average time to read a single row over the network (in microseconds) \\ \hline
		\texttt{writelatency}					&	bigint		&	The average time to write a single row over the network (in microseconds) \\ \hline
		\texttt{openscannerlatency} 	&	bigint		&	The average time to open a remote scanner against this partition (in microseconds) \\ \hline
		\texttt{closescannerlatency}	&	bigint		&	The average time to close a remote scanner against this partition (in microseconds) \\ \hline
	\end{tabular}
	\caption{\texttt{\systablestats table schema}}
	\label{table:tableStats}
\end{table}

\begin{table}
	\begin{tabular}{|l|c|p{6cm}|}
		\hline
		\bf{Name}											&	\bf{Type}	& \bf{Description} \\ \hline
		\texttt{schemaname}						&	varchar		&	The name of the schema for this conglomerate \\ \hline
		\texttt{tablename}						&	varchar		&	The name of the table for this conglomerate \\ \hline
		\texttt{total\_row\_count}		&	bigint		&	The total number of rows in the conglomerate \\ \hline
		\texttt{avg\_row\_count}			&	bigint		&	The average number of rows in a single partition \\ \hline
		\texttt{total\_size}					&	bigint		&	The total number of bytes in the conglomerate \\ \hline
		\texttt{num\_partitions}			&	bigint		&	The number of active partitions for this conglomerate \\ \hline
		\texttt{avg\_partition\_size}	&	bigint		&	The average size of a single partition (in bytes) \\ \hline
		\texttt{row\_width}						&	bigint		&	The average width of a single row (in bytes) \\ \hline
		\texttt{total\_query\_count}	&	bigint		&	The total number of queries against any partition \\ \hline
		\texttt{avg\_query\_count}		& bigint		&	The average number of queries addressed to a single partition \\ \hline
		\texttt{avg\_local\_read\_latency}	&	bigint	&	The average local read latency, across all partitions (in microseconds) \\ \hline
		\texttt{avg\_remote\_read\_latency}	&	bigint	&	The average remote read latency, across all partitions (in microseconds) \\ \hline
		\texttt{avg\_write\_latency} 				&	bigint	&	the average write latency, across all partitions (in microseconds) \\ \hline
	\end{tabular}
\caption{\systablestatistics \texttt{table schema}}
\label{table:columnStats}
\end{table}

\subsection{Column Statistics}
\syscolumnstats stores the per-partition view of column data, particularly information about cardinality and distributions of column values; its schema can be found in Table-\ref{table:columnStats}. \syscolumnstatistics is used to present a human-readable view over \syscolumnstats, with a schema found in Table-\ref{table:columnStatistics}. 

\begin{table}
	\begin{tabular}{|l|c|p{6cm}|}
		\hline
		\bf{Name}											&	\bf{Type}	& \bf{Description} \\ \hline
		\texttt{conglom\_id}					&	bigint		&	the unique id for the owning conglomerate \\ \hline
		\texttt{partition\_id}				&	varchar		&	the unique id for the owning partition \\ \hline
		\texttt{column\_id}						&	integer		&	the unique id for the column \\ \hline
		\texttt{data}									&	binary		&	A binary-encoded representation of column data \\ \hline
	\end{tabular}
	\caption{\syscolumnstats \texttt{table schema}}
	\label{table:columnStats}
\end{table}

\begin{table}
	\begin{tabular}{|l|c|p{6cm}|}
		\hline
		\bf{Name}											&	\bf{Type}	& \bf{Description} \\ \hline
		\texttt{schemaname}						&	varchar		&	The name of the schema for this conglomerate \\ \hline
		\texttt{tablename}						&	varchar		&	The name of the table for this conglomerate \\ \hline
		\texttt{columnname}						&	varchar		&	The name of the column \\ \hline
		\texttt{cardinality}					&	bigint		&	The cardinality of the column \\ \hline
		\texttt{null\_count}					&	bigint		&	The number of null values for the column \\ \hline
		\texttt{null\_fraction}				&	real			&	The ratio of null to non-null records \\ \hline
		\texttt{min\_value}						&	varchar		&	A string version of the minimum value of the column \\ \hline
		\texttt{max\_value}						&	varchar		&	A string version of the maximum value of the column \\ \hline
		\texttt{top\_k}								&	varchar		&	A comma-separated list of $(value,count,error)$ tuples representing the top $k$ most frequent non-null values for this column. \\ \hline
	\end{tabular}
	\caption{\syscolumnstatistics \texttt{table schema}}
	\label{table:columnStatistics}
\end{table}

